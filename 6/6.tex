\documentclass[UTF8,12pt]{ctexart}
\usepackage{amsmath}
\usepackage{graphicx}
\usepackage{physics}
\usepackage{fontspec} % XeLaTeX 字体支持
\usepackage{listings}
\usepackage{xcolor}

% 等宽字体(可根据系统调整)
\setmonofont{DejaVu Sans Mono}

% 取消段首缩进
\setlength{\parindent}{0pt}

% listings 配置
\lstset{
	basicstyle=\ttfamily\tiny,
	keywordstyle=\color{blue}\bfseries,
	commentstyle=\color{green!50!black}\itshape,
	stringstyle=\color{purple},
	numbers=left,
	numberstyle=\tiny,
	breaklines=true,
	tabsize=2,
	frame=single,
	captionpos=b,
	columns=fullflexible,
	showstringspaces=false
}

\begin{document}

1.
空间3R肘机械臂各齐次变换矩阵有:
\[
\begin{aligned}
    {}^0 T_1 &= \begin{bmatrix}
                    \cos\theta_1 & 0 & -\sin\theta_1 & 0 \\
                    \sin\theta_1 & 0 & \cos\theta_1 & 0 \\
                    0 & -1 & 0 & d_1 \\
                    0 & 0 & 0 & 1
                \end{bmatrix} \\
    {}^1 T_2 &= \begin{bmatrix}
                    \cos\theta_2 & -\sin\theta_2 & 0 & a_2\cos\theta_2 \\
                    \sin\theta_2 & \cos\theta_2 & 0 & a_2\sin\theta_2 \\
                    0 & 0 & 1 & 0 \\
                    0 & 0 & 0 & 1
                \end{bmatrix} \\
    {}^2 T_3 &= \begin{bmatrix}
                    \cos\theta_3 & -\sin\theta_3 & 0 & a_3\cos\theta_3 \\
                    \sin\theta_3 & \cos\theta_3 & 0 & a_3\sin\theta_3 \\
                    0 & 0 & 1 & 0 \\
                    0 & 0 & 0 & 1
                \end{bmatrix}
\end{aligned}
\]

故位置级正运动学方程为:
\[
{}^0 T_3 = {}^0 T_1 \cdot {}^1 T_2 \cdot {}^2 T_3
        = \begin{bmatrix}
            \cos\theta_1 \cos(\theta_2 + \theta_3) & -\cos\theta_1 \sin(\theta_2 + \theta_3) & -\sin\theta_1 & x \\
            \sin\theta_1 \cos(\theta_2 + \theta_3) & -\sin\theta_1 \sin(\theta_2 + \theta_3) & \cos\theta_1 & y \\
            -\sin(\theta_2 + \theta_3) & -\cos(\theta_2 + \theta_3) & 0 & z \\
            0 & 0 & 0 & 1
        \end{bmatrix}
\]

其中位置向量为:
\[
    {}^0 P_3 = \begin{bmatrix}
                    x \\
                    y \\
                    z
                \end{bmatrix}
            = \begin{bmatrix}
                \cos\theta_1 (a_2 \cos\theta_2 + a_3 \cos(\theta_2 + \theta_3)) \\
                \sin\theta_1 (a_2 \cos\theta_2 + a_3 \cos(\theta_2 + \theta_3)) \\
                d_1 - a_2 \sin\theta_2 - a_3 \sin(\theta_2 + \theta_3)
            \end{bmatrix}
\]

\vspace{3em}


下面计算圆轨迹参数方程:

由圆心及轨迹上点的坐标计算圆半径为:
\[
r = \|P_0 - O_c\| = \sqrt{0.2^2 + 0.1^2 + 0.2^2} = 0.3
\]

构建圆弧的局部坐标系。设 \(x\) 沿 \(P_0 - O_c\) 方向:
\[
i = \frac{P_0 - O_c}{\|P_0 - O_c\|} 
    = \begin{bmatrix}
        0.6667 \\ 0.3333 \\ 0.6667
    \end{bmatrix}
\]

为确定圆弧平面上的 \(j\) 矢量,利用 \(P_f\) 构造圆弧平面的法向量:
\[
n = (P_0 - O_c) \times (P_f - O_c) 
    = \begin{bmatrix}
        0.2 \\ 0.1 \\ 0.2
    \end{bmatrix} \times \begin{bmatrix}
        0.1 \\ -0.2 \\ -0.2
    \end{bmatrix}
    = \begin{bmatrix}
        0.02 \\ 0.06 \\ -0.05
    \end{bmatrix}
\]

归一化得到 \(k\) 矢量(垂直于圆弧平面):
\[
k = \frac{n}{\|n\|} 
    = \frac{1}{\sqrt{0.02^2 + 0.06^2 + 0.05^2}} \begin{bmatrix}
        0.02 \\ 0.06 \\ -0.05
    \end{bmatrix}
    = \begin{bmatrix}
        0.2481 \\ 0.7442 \\ -0.6202
    \end{bmatrix}
\]

最后由右手系法则得:
\[
j = k \times i 
    = \begin{bmatrix}
        0.7029 \\ -0.5789 \\ -0.4136
    \end{bmatrix}
\]

从 \(P_0\) 到 \(P_f\) 的圆心角为:
\[
\phi_f = \arccos\left(\frac{(P_0 - O_c) \cdot (P_f - O_c)}{\|P_0 - O_c\| \|P_f - O_c\|}\right) = 116.39^\circ
\]

圆轨迹参数方程为:
\[
P(\lambda) = O_c + r(\cos(\phi_0 + \lambda (\phi_f - \phi_0)) \cdot i + \sin(\phi_0 + \lambda (\phi_f - \phi_0)) \cdot j) \quad \lambda \in [0, 1]
\]

\vspace{3em}


下面将参数时序化:

采用三次多项式进行时间规划:
\[
\lambda(t) = a_0 + a_1 t + a_2 t^2 + a_3 t^3
\]

满足边界条件:\(\lambda(0)=0\),\(\dot{\lambda}(0)=0\),\(\lambda(t_f)=1\),\(\dot{\lambda}(t_f)=0\)。
确定为:
\[
\lambda(t) = 3\left(\frac{t}{t_f}\right)^2 - 2\left(\frac{t}{t_f}\right)^3
\]


\vspace{3em}

附Matlab程序:
\lstinputlisting[language=Matlab]{src/1.m}

\end{document}
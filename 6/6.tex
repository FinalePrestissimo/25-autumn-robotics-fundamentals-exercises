\documentclass[UTF8,12pt]{ctexart}
\usepackage{amsmath}
\usepackage{amssymb}
\usepackage{graphicx}
\usepackage{physics}
\usepackage{xeCJK} % XeLaTeX CJK support (loads fontspec)
\usepackage{listings}
\usepackage{xcolor}
\usepackage{float} % 支持 [H] 强制浮动体位置
\usepackage{caption}

% 等宽字体(可根据系统调整)
\setmonofont{DejaVu Sans Mono}

% 取消段首缩进
\setlength{\parindent}{0pt}

% listings 配置
\lstset{
	basicstyle=\ttfamily\tiny,
	keywordstyle=\color{blue}\bfseries,
	commentstyle=\color{green!50!black}\itshape,
	stringstyle=\color{purple},
	numbers=left,
	numberstyle=\tiny,
	breaklines=true,
	tabsize=2,
	frame=single,
	captionpos=b,
	columns=fullflexible,
	showstringspaces=false
}

\begin{document}

1.
空间3R肘机械臂各齐次变换矩阵有:
\[
\begin{aligned}
    {}^0 T_1 &= \begin{bmatrix}
                    \cos\theta_1 & 0 & -\sin\theta_1 & 0 \\
                    \sin\theta_1 & 0 & \cos\theta_1 & 0 \\
                    0 & -1 & 0 & d_1 \\
                    0 & 0 & 0 & 1
                \end{bmatrix} \\
    {}^1 T_2 &= \begin{bmatrix}
                    \cos\theta_2 & -\sin\theta_2 & 0 & a_2\cos\theta_2 \\
                    \sin\theta_2 & \cos\theta_2 & 0 & a_2\sin\theta_2 \\
                    0 & 0 & 1 & 0 \\
                    0 & 0 & 0 & 1
                \end{bmatrix} \\
    {}^2 T_3 &= \begin{bmatrix}
                    \cos\theta_3 & -\sin\theta_3 & 0 & a_3\cos\theta_3 \\
                    \sin\theta_3 & \cos\theta_3 & 0 & a_3\sin\theta_3 \\
                    0 & 0 & 1 & 0 \\
                    0 & 0 & 0 & 1
                \end{bmatrix}
\end{aligned}
\]

故位置级正运动学方程为:
\[
{}^0 T_3 = {}^0 T_1 \cdot {}^1 T_2 \cdot {}^2 T_3
        = \begin{bmatrix}
            \cos\theta_1 \cos(\theta_2 + \theta_3) & -\cos\theta_1 \sin(\theta_2 + \theta_3) & -\sin\theta_1 & x \\
            \sin\theta_1 \cos(\theta_2 + \theta_3) & -\sin\theta_1 \sin(\theta_2 + \theta_3) & \cos\theta_1 & y \\
            -\sin(\theta_2 + \theta_3) & -\cos(\theta_2 + \theta_3) & 0 & z \\
            0 & 0 & 0 & 1
        \end{bmatrix}
\]

其中位置向量为:
\[
    {}^0 P_3 = \begin{bmatrix}
                    x \\
                    y \\
                    z
                \end{bmatrix}
            = \begin{bmatrix}
                \cos\theta_1 (a_2 \cos\theta_2 + a_3 \cos(\theta_2 + \theta_3)) \\
                \sin\theta_1 (a_2 \cos\theta_2 + a_3 \cos(\theta_2 + \theta_3)) \\
                d_1 - a_2 \sin\theta_2 - a_3 \sin(\theta_2 + \theta_3)
            \end{bmatrix}
\]

\vspace{3em}


下面计算圆轨迹参数方程:

由圆心及轨迹上点的坐标计算圆半径为:
\[
r = \|P_0 - O_c\| = \sqrt{0.2^2 + 0.1^2 + 0.2^2} = 0.3
\]

构建圆弧的局部坐标系。设 \(x\) 沿 \(P_0 - O_c\) 方向:
\[
i = \frac{P_0 - O_c}{\|P_0 - O_c\|} 
    = \begin{bmatrix}
        0.6667 \\ 0.3333 \\ 0.6667
    \end{bmatrix}
\]

为确定圆弧平面上的 \(j\) 矢量,利用 \(P_f\) 构造圆弧平面的法向量:
\[
n = (P_0 - O_c) \times (P_f - O_c) 
    = \begin{bmatrix}
        0.2 \\ 0.1 \\ 0.2
    \end{bmatrix} \times \begin{bmatrix}
        0.1 \\ -0.2 \\ -0.2
    \end{bmatrix}
    = \begin{bmatrix}
        0.02 \\ 0.06 \\ -0.05
    \end{bmatrix}
\]

归一化得到 \(k\) 矢量(垂直于圆弧平面):
\[
k = \frac{n}{\|n\|} 
    = \frac{1}{\sqrt{0.02^2 + 0.06^2 + 0.05^2}} \begin{bmatrix}
        0.02 \\ 0.06 \\ -0.05
    \end{bmatrix}
    = \begin{bmatrix}
        0.2481 \\ 0.7442 \\ -0.6202
    \end{bmatrix}
\]

最后由右手系法则得:
\[
j = k \times i 
    = \begin{bmatrix}
        0.7029 \\ -0.5788 \\ -0.4134
    \end{bmatrix}
\]

从 \(P_0\) 到 \(P_f\) 的圆心角为:
\[
\phi_f = \arccos\left(\frac{(P_0 - O_c) \cdot (P_f - O_c)}{\|P_0 - O_c\| \|P_f - O_c\|}\right) = 116.39^\circ
\]

圆轨迹参数方程为:
\[
P(\lambda) = O_c + r(\cos(\phi_0 + \lambda (\phi_f - \phi_0)) \cdot i + \sin(\phi_0 + \lambda (\phi_f - \phi_0)) \cdot j) \quad \lambda \in [0, 1]
\]

\vspace{3em}


下面将参数时序化:

采用三次多项式进行时间规划:
\[
\lambda(t) = a_0 + a_1 t + a_2 t^2 + a_3 t^3
\]

满足边界条件:\(\lambda(0)=0\),\(\dot{\lambda}(0)=0\),\(\lambda(t_f)=1\),\(\dot{\lambda}(t_f)=0\)。
确定为:
\[
\lambda(t) = 3\left(\frac{t}{t_f}\right)^{2} - 2\left(\frac{t}{t_f}\right)^{3}
\]

\vspace{3em}


求解得机器人关节角,末端位置曲线及3D轨迹如下:
\begin{figure}[H]
    \centering
    \begin{minipage}[c]{0.45\textwidth}
        \centering
        \includegraphics[width=\textwidth]{asset/joint.PNG}
        
        \vspace{1em}
        
        \includegraphics[width=\textwidth]{asset/end.PNG}
    \end{minipage}
    \hfill
    \begin{minipage}[c]{0.5\textwidth}
        \centering
        \includegraphics[width=\textwidth]{asset/3D.PNG}
    \end{minipage}
\end{figure}

\vspace{3em}


附Matlab程序:
\lstinputlisting[language=Matlab]{src/q1.m}

\newpage


2.
空间3R球腕机械臂各齐次变换矩阵有:
\[
\begin{aligned}
    {}^0 T_1 &= \begin{bmatrix}
                    \cos\theta_1 & 0 & -\sin\theta_1 & 0 \\
                    \sin\theta_1 & 0 & \cos\theta_1 & 0 \\
                    0 & -1 & 0 & d_1 \\
                    0 & 0 & 0 & 1
                \end{bmatrix} \\
    {}^1 T_2 &= \begin{bmatrix}
                    \cos\theta_2 & 0 & -\sin\theta_2 & 0 \\
                    \sin\theta_2 & 0 & \cos\theta_2 & 0 \\
                    0 & -1 & 0 & 0 \\
                    0 & 0 & 0 & 1
                \end{bmatrix} \\
    {}^2 T_3 &= \begin{bmatrix}
                    \cos\theta_3 & -\sin\theta_3 & 0 & 0 \\
                    \sin\theta_3 & \cos\theta_3 & 0 & 0 \\
                    0 & 0 & 1 & d_3 \\
                    0 & 0 & 0 & 1
                \end{bmatrix}
\end{aligned}
\]

故姿态级正运动学方程为:
\[
{}^0 T_3 = {}^0 T_1 \cdot {}^1 T_2 \cdot {}^2 T_3
        = \begin{bmatrix}
            {}^0 R_3 & {}^0 P_3 \\
            0 & 1
        \end{bmatrix}
\]

其中旋转矩阵为:
\[
    {}^0 R_3 = \begin{bmatrix}
                    \cos\theta_1 \cos\theta_2 \cos\theta_3 + \sin\theta_1 \sin\theta_3 & -\cos\theta_1 \cos\theta_2 \sin\theta_3 + \sin\theta_1 \cos\theta_3 & -\cos\theta_1 \sin\theta_2 \\
                    \sin\theta_1 \cos\theta_2 \cos\theta_3 - \cos\theta_1 \sin\theta_3 & -\sin\theta_1 \cos\theta_2 \sin\theta_3 - \cos\theta_1 \cos\theta_3 & -\sin\theta_1 \sin\theta_2 \\
                    -\sin\theta_2 \cos\theta_3 & \sin\theta_2 \sin\theta_3 & -\cos\theta_2
                \end{bmatrix}
\]

位置向量为:
\[
    {}^0 P_3 = \begin{bmatrix}
                    -d_3 \cos\theta_1 \sin\theta_2 \\
                    -d_3 \sin\theta_1 \sin\theta_2 \\
                    d_1 - d_3 \cos\theta_2
                \end{bmatrix}
\]

\vspace{3em}


(1)
采用三次多项式进行时间规划:
\[
s(t) = a_0 + a_1 t + a_2 t^2 + a_3 t^3
\]

满足边界条件:\(s(0)=0\),\(\dot{s}(0)=0\),\(s(t_f)=1\),\(\dot{s}(t_f)=0\),
其中 \(t_f = 10\) s。求解得:
\[
\begin{cases}
a_0 = 0 \\
a_1 = 0 \\
a_2 = \frac{3}{t_f^2} \\
a_3 = -\frac{2}{t_f^3}
\end{cases}
\]

因此归一化参数为:
\[
s(t) = 3\left(\frac{t}{t_f}\right)^2 - 2\left(\frac{t}{t_f}\right)^3
\]

对姿态角各分量进行插值:
\[
\Phi(t) = \Phi_0 + (\Phi_f - \Phi_0) \cdot s(t)
\]

其中初始姿态角 \(\Phi_0 = [-157.8240^\circ, 46.0418^\circ, 70.4798^\circ]^T\),
终止姿态角 \(\Phi_f = [174.9616^\circ, 8.6492^\circ, 90.3813^\circ]^T\)。

\vspace{5em}


(2) 
首先计算初始和终止姿态对应的旋转矩阵。xyz动轴欧拉角对应的旋转矩阵为:
\[
R = R_x(\phi) R_y(\theta) R_z(\psi)
\]

计算得:
\[
R_0 = R_x(-157.8240^\circ) R_y(46.0418^\circ) R_z(70.4798^\circ)
\]
\[
R_f = R_x(174.9616^\circ) R_y(8.6492^\circ) R_z(90.3813^\circ)
\]

计算相对旋转矩阵:
\[
R_{rel} = R_0^T R_f
\]

从旋转矩阵转换至轴-角有:
\[
\begin{aligned}
    \theta_{eq} &= \arccos\left(\frac{\text{tr}(R_{rel}) - 1}{2}\right) \\
    \mathbf{k} &= \frac{1}{2\sin\theta_{eq}} \begin{bmatrix}
                                                R_{32} - R_{23} \\
                                                R_{13} - R_{31} \\
                                                R_{21} - R_{12}
                                            \end{bmatrix}
\end{aligned}
\]

采用五次多项式进行时间规划:
\[
s(t) = a_0 + a_1 t + a_2 t^2 + a_3 t^3 + a_4 t^4 + a_5 t^5
\]

满足边界条件:\(s(0)=0\),\(\dot{s}(0)=0\),\(\ddot{s}(0)=0\),\(s(t_f)=1\),\(\dot{s}(t_f)=0\),\(\ddot{s}(t_f)=0\)。

求解得:
\[
\begin{cases}
a_0 = a_1 = a_2 = 0 \\
a_3 = \frac{10}{t_f^3} \\
a_4 = -\frac{15}{t_f^4} \\
a_5 = \frac{6}{t_f^5}
\end{cases}
\]

因此:
\[
s(t) = 10\left(\frac{t}{t_f}\right)^3 - 15\left(\frac{t}{t_f}\right)^4 + 6\left(\frac{t}{t_f}\right)^5
\]

对等效转角进行插值:
\[
\theta(t) = \theta_{eq} \cdot s(t)
\]

则每个 \(t\) 时刻相对旋转矩阵为:
\[
R(t) = I + \sin\theta(t) \cdot K + (1-\cos\theta(t)) \cdot K^2
\]

其中 \(K\) 为等效转轴 \(\mathbf{k} = [k_x, k_y, k_z]^T\) 对应的反对称矩阵(斜对称矩阵),定义为:
\[
K = \mathbf{k}^\times = \begin{bmatrix}
0 & -k_z & k_y \\
k_z & 0 & -k_x \\
-k_y & k_x & 0
\end{bmatrix}
\]

\vspace{3em}


两种规划方法对比如下:

\textbf{方法1(三次多项式直接插值):}
\begin{itemize}
    \item 优点:实现简单,计算量小
    \item 缺点:各姿态角分量独立插值,可能导致非最优路径旋转;三次多项式使角速度在端点虽然为零,但中间可能不够平滑
\end{itemize}

\textbf{方法2(五次多项式等效转角插值):}
\begin{itemize}
    \item 优点:基于等效转轴和转角,保证最短路径旋转;五次多项式使加速度在起止点为零,运动更加平滑;物理意义明确
    \item 缺点:计算稍复杂
\end{itemize}

\vspace{2em}


求解得到的姿态角曲线、角速度曲线及关节角曲线如下图所示:

\begin{figure}[H]
    \centering
    \includegraphics[width=1.\textwidth]{asset/end_euler.jpg}
    \caption*{末端姿态角变化曲线}
\end{figure}

\begin{figure}[H]
    \centering
    \includegraphics[width=1.\textwidth]{asset/end_angvel.jpg}
    \caption*{末端角速度曲线}
\end{figure}

\begin{figure}[H]
    \centering
    \includegraphics[width=1.\textwidth]{asset/joint_ang.jpg}
    \caption*{关节角曲线}
\end{figure}

\vspace{3em}


% 附Matlab程序:
% \lstinputlisting[language=Matlab]{src/2.m}

\newpage


3.
\begin{table}[H]
\centering
\begin{tabular}{cl}
\hline
变量 & 说明 \\
\hline
\(m\) & 机械臂旋转关节总数 \\
\(n\) & 末端自由度(\(n \leq 6\)) \\
\(\boldsymbol{\tau}\) & 关节力矩向量 \\
\(\mathbf{q}\) & 关节角向量 \\
\(\delta\mathbf{q}\) & 关节虚位移 \\
\(\mathbf{F} \in \mathbb{R}^n\) & 末端广义力向量(包含力和力矩) \\
\(\mathbf{x}_e \in \mathbb{R}^n\) & 末端广义位移向量(包含线位移和角位移) \\
\(\delta\mathbf{x}_e \in \mathbb{R}^n\) & 末端广义虚位移 \\
\hline
\end{tabular}
\end{table}

\vspace{3em}


根据虚功原理,系统处于静力平衡时,所有外力在虚位移上所做的虚功之和为零。

关节空间的虚功为:
\[
\delta W_{\text{joint}} = \boldsymbol{\tau}^T \delta\mathbf{q}
\]

末端操作空间的虚功为:
\[
\delta W_{\text{end}} = \mathbf{F}^T \delta\mathbf{x}_e
\]

\vspace{1em}

末端广义虚位移与关节虚位移的关系由雅可比矩阵 \(\mathbf{J}(\mathbf{q}) \in \mathbb{R}^{n \times m}\) 给出:
\[
\delta\mathbf{x}_e = \mathbf{J}(\mathbf{q}) \delta\mathbf{q}
\]

其中雅可比矩阵定义为:
\[
\mathbf{J}(\mathbf{q}) = \frac{\partial \mathbf{x}_e}{\partial \mathbf{q}}
\]

\vspace{1em}

将末端虚位移关系代入末端虚功表达式:
\[
\delta W_{\text{end}} = \mathbf{F}^T \mathbf{J}(\mathbf{q}) \delta\mathbf{q}
\]

\vspace{1em}

根据虚功原理,系统总虚功为零:
\[
\delta W = \delta W_{\text{joint}} - \delta W_{\text{end}} = 0
\]

代入得:
\[
\boldsymbol{\tau}^T \delta\mathbf{q} - \mathbf{F}^T \mathbf{J}(\mathbf{q}) \delta\mathbf{q} = 0
\]

整理为:
\[
\left(\boldsymbol{\tau}^T - \mathbf{F}^T \mathbf{J}(\mathbf{q})\right) \delta\mathbf{q} = 0
\]

由于虚位移 \(\delta\mathbf{q}\) 是任意的,上式恒成立当且仅当:
\[
\boldsymbol{\tau}^T - \mathbf{F}^T \mathbf{J}(\mathbf{q}) = \mathbf{0}
\]

即静力学方程为:
\[
\boldsymbol{\tau} = \mathbf{J}^T(\mathbf{q}) \mathbf{F}
\]


\end{document}